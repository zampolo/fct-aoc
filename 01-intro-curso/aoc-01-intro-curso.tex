\documentclass[
size=17pt,
paper=smartboard,
mode=present,
display=slidesnotes,
style=paintings,
nopagebreaks,
blackslide,
fleqn]{powerdot}

% styles: sailor, paintings
% wj capsules prettybox
% mode = handout or present


\usepackage{amsmath,graphicx,color,amsfonts}
\usepackage[brazilian]{babel}
\usepackage[utf8]{inputenc}
\newcommand{\palette}{Europa}


% palettes:
%    - sailor: Sea, River, Wine, Chocolate, Cocktail 
%    - paintings: Syndics, Skater, GoldenGate, Moitessier, PearlEarring, Lamentation, HolyWood, Europa, MayThird, Charon 

\newcommand{\cursopequeno}{EC01008 AOC}
\newcommand{\cursogrande}{\Large EC01008 -- Arquitetura e organização de computadores}



\author{Ronaldo de Freitas Zampolo\\FCT-ITEC-UFPA}
\date{2023-4}



\pdsetup{
   lf = {\cursopequeno},
   cf = {\theslide},
   rf = {Apresentação do curso}, palette = {\palette}, randomdots={false}
}


%opening
\title{\cursogrande\\ \vspace{1cm}Apresentação do curso}

\begin{document}
   
   \maketitle[randomdots={false}]
   \begin{slide}{Agenda}
      \tableofcontents[content=sections]
   \end{slide}

   \section[ slide = true]{Professor e atendimento}
      \begin{slide}[toc=]{Professor e atendimento}
         \begin{itemize}
            \item Professor: Ronaldo de Freitas Zampolo
            \item Afiliação:\\
                  Laboratório de Processamento de Sinais - LaPS\\
                  Faculdade de Eng. da Computação e Telecomunicações - FCT\\
                  Instituto de Tecnologia - ITEC\\
                  Universidade Federal do Pará - UFPA\\
            %\item Plantão de dúvidas:\\
            %      Segunda-feira: 13h00 - 14h00\\
            %      Chat do SIGAA\\
                  \texttt{zampolo@ufpa.br}\\ 
                  %\texttt{zampolo@ieee.org}\\
            %      \texttt{www.laps.ufpa.br/zampolo}
         \end{itemize}
      \end{slide}
      
   \section[ slide = true]{Características do Curso}
      \begin{slide}[toc=]{Características do Curso}
         \begin{itemize}
            \item Carga horária: 60h
            \item Atividade síncrona: segundas e sextas às 7h30 
            \item Tópicos:
            \begin{itemize}
               \item Evolução e desempenho do computador
               \item Visão de alto nível da função e interconexão do computador
               \item Memória
               \item Entrada e saída
               \item Conjuntos de instruções
               \item Estrutura e função do processador
               %\item Máquinas RISC
               %\item Paralelismo em nível de instrução
               %\item Máquinas superescalares
               %\item Unidade de controle
               %\item Processamento paralelo
            \end{itemize}
         \end{itemize}         
      \end{slide}
      
   \section[ slide = true]{Bibliografia}
      \begin{slide}[toc=]{Bibliografia}
         \begin{itemize}
            \item Bibliografia básica
            \begin{itemize}
               \item \textbf{W. Stallings, \emph{Arquitetura e Organização de Computadores}, 8ª edição, Pearson,  2010.}\\ \texttt{www.prenhall.com/stallings\_br}\\ \texttt{WilliamStallings.com/StudentSupport.html}
               \item \textbf{W. Stallings, \emph{Computer organization and architecture -- designing for performance}, tenth edition, Pearson,  2016.}
            \end{itemize}
            \item Bibliografia complementar
            \begin{itemize}
               \item {N. Nisan and S. Schocken, \emph{The Elements of Computing Systems: Building a Modern Computer from First Principles}, 2nd ed., The MIT Press, 2021}
	       \item \textbf{J. L. Hennessy e D. A. Patterson, \emph{Arquitetura de computadores -- uma abordagem quantitativa}, sexta edição, LTC, 2019.}
               \item {A. S. Tanenbaum, T. Austin, \emph{Structured computer organization}, 6th ed., Pearson, 2013.} 
               \item {A. S. Tanenbaum, \emph{Organização estruturada de computadores}, 5ª ed., Pearson, 2007.} 
            \end{itemize}
         \end{itemize}
      \end{slide}

    \section[slide=true]{Habilidades e competências}
      \begin{slide}[toc=]{Habilidades competências}
         \begin{itemize}
           % \item Objetivos:
%		    \begin{itemize}
%			    \item Conhecer os principais elementos constuintes de um sistema computacional
%			    \item Compreender como diferentes opções de implementação impactam no custo e no desempenho de um sistema computacional
%			    \item Entender o funcionamento básico de um sistema computacional
%		    \end{itemize}
	     \item Habilidades e competências:
		     \begin{itemize}
			     \item Conhecimento sobre os \textbf{elementos principais de um sistema computacional}, suas funções e como se relacionam;
			     \item \textbf{Análise crítica de soluções técnicas} concorrentes para implementação de sistemas computacionais, considerando requisitos de desempenho e custo de um projeto;
			     \item \textbf{Diferenciação entre abordagens em software} (uso de funções, tradução vs. interpretação, assembly vs. linguagem de nível mais alto) de acordo com requisitos de desempenho de execução e custo de implementação;
			     \item Compreensão dos princípios de funcionamento das \textbf{estratégias utilizadas para maximizar o desempenho} de sistemas computacionais.
	            \end{itemize}
         \end{itemize}
      \end{slide}

    \section[ slide = true]{Metodologia, ferramentas e avaliação}
      \begin{slide}[toc=]{Metodologia, ferramentas e avaliação}
         \begin{itemize}
		 \item Metodologia utilizada: aula invertida (nem sempre) 
		    \begin{itemize}
			    \item Em sala de aula: exercícios, aprofundamento, testes.
			    \item Fora da sala de aula: leituras, vídeos, pesquisas, trabalhos. 
		    \end{itemize}
	    \item Ferramentas usadas:
		    \begin{itemize}
			    \item Google Meet % e SIGAA (Chat)
			    \item SIGAA: repositório e sistema de entrega de material
			   % \item Google Colaboratory (Python): simulações, programação 
			    \item Simuladores: download em https://www.nand2tetris.org/software
		    \end{itemize}

            \item Avaliação continuada
            \begin{itemize}
		    \item Trabalhos ($Tr$) 
		    \item Tarefas ($Ta$)
		    \item Provas ($Pr$)
	    \end{itemize}
	 \end{itemize}
      \end{slide}
      
      \begin{slide}[toc=]{Datas dos testes escritos}
         \begin{itemize}
       	    \item Cálculo da nota ($N$):
            \begin{equation*}
               N=\frac{( Pr \times 4 + Tr \times 4 + Ta \times 2 )} {10}
            \end{equation*}
          \item Tabela de mapeamento:
		  \begin{itemize}
			  \item 0,0 -- 4,9:  INS
			  \item 5,0 -- 6,9:  REG
			  \item 7,0 -- 8,9:  BOM
			  \item 9,0 -- 10,0:  EXC
		  \end{itemize}
            \item Datas prováveis das provas:
		    \begin{itemize}
			    \item Prova 1: 22/setembro
			    \item Prova 2: 06/novembro
			    \item Prova 3: 15/dezembro
		    \end{itemize}
         \end{itemize}
      \end{slide}

\end{document}
