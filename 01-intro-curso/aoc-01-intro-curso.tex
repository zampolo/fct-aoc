\documentclass[
size=17pt,
paper=smartboard,
mode=present,
display=slidesnotes,
style=paintings,
nopagebreaks,
blackslide,
fleqn]{powerdot}

% styles: sailor, paintings
% wj capsules prettybox
% mode = handout or present


\usepackage{amsmath,graphicx,color,amsfonts}
\usepackage[brazilian]{babel}
\usepackage[utf8]{inputenc}
\newcommand{\palette}{Europa}


% palettes:
%    - sailor: Sea, River, Wine, Chocolate, Cocktail 
%    - paintings: Syndics, Skater, GoldenGate, Moitessier, PearlEarring, Lamentation, HolyWood, Europa, MayThird, Charon 

\newcommand{\cursopequeno}{EC01008 AOC}
\newcommand{\cursogrande}{\Large EC01008 -- Arquitetura e organização de computadores}



\author{Ronaldo de Freitas Zampolo\\FCT-ITEC-UFPA}
\date{2023-4}



\pdsetup{
   lf = {\cursopequeno},
   rf = {Apresentação do curso}, palette = {\palette}, randomdots={false}
}


%opening
\title{\cursogrande\\ \vspace{1cm}Apresentação do curso}

\begin{document}
   
   \maketitle[randomdots={false}]
   \begin{slide}{Agenda}
      \tableofcontents[content=sections]
   \end{slide}

   \section[ slide = true]{Professor e atendimento}
      \begin{slide}[toc=]{Professor e atendimento}
         \begin{itemize}
            \item Professor: Ronaldo de Freitas Zampolo
            \item Afiliação:\\
                  Laboratório de Processamento de Sinais - LaPS\\
                  Faculdade de Eng. da Computação e Telecomunicações - FCT\\
                  Instituto de Tecnologia - ITEC\\
                  Universidade Federal do Pará - UFPA\\
            %\item Plantão de dúvidas:\\
            %      Segunda-feira: 13h00 - 14h00\\
            %      Chat do SIGAA\\
                  \texttt{zampolo@ufpa.br}\\ 
                  %\texttt{zampolo@ieee.org}\\
            %      \texttt{www.laps.ufpa.br/zampolo}
         \end{itemize}
      \end{slide}
      
   \section[ slide = true]{Características do Curso}
      \begin{slide}[toc=]{Características do Curso}
         \begin{itemize}
            \item Carga horária: 60 h
            \item Atividade síncrona: terças às 7h30
            \item Tópicos:
            \begin{itemize}
               \item Evolução e desempenho do computador
               \item Visão de alto nível da função e interconexão do computador
               \item Memória
               \item Entrada e saída
               \item Conjuntos de instruções
               \item Estrutura e função do processador
               %\item Máquinas RISC
               \item Paralelismo em nível de instrução
               \item Máquinas superescalares
               %\item Unidade de controle
               %\item Processamento paralelo
            \end{itemize}
         \end{itemize}         
      \end{slide}
      
   \section[ slide = true]{Bibliografia}
      \begin{slide}[toc=]{Bibliografia}
         \begin{itemize}
            \item Bibliografia básica
            \begin{itemize}
               \item \textbf{W. Stallings, \emph{Arquitetura e Organização de Computadores}, 8ª edição, Pearson,  2010.}\\ \texttt{www.prenhall.com/stallings\_br}\\ \texttt{WilliamStallings.com/StudentSupport.html}
               \item \textbf{W. Stallings, \emph{Computer organization and architecture -- designing for performance}, 10th edition, Pearson,  2013.}
            \end{itemize}
            \item Bibliografia complementar
            \begin{itemize}
               \item {A. S. Tanenbaum, T. Austin, \emph{Structured computer organization}, 6th ed., Pearson, 2013.} 
               \item {A. S. Tanenbaum, \emph{Organização estruturada de computadores}, 5ª ed., Pearson, 2007.} 
               \item J. L. Hennessy e D. A. Patterson, \emph{Arquitetura de Computadores -- Uma abordagem quantitativa}, Editora Campus, 2003.
            \end{itemize}
         \end{itemize}
      \end{slide}

    \section[slide=true]{Objetivos, habilidades e competências}
      \begin{slide}[toc=]{Objetivos, habilidades competências}
         \begin{itemize}
            \item Objetivos:
		    \begin{itemize}
			    \item Conhecer os principais elementos consituintes de um sistema computacional
			    \item Compreender como diferentes opções de implementação impactam no custo e no desempenho de um sistema computacional
			    \item Entender o funcionamento básico de um sistema computacional
		    \end{itemize}
	     \item Habilidades e competências:
		     \begin{itemize}
			     \item Conhecimento sobre os elementos principais de um sistema computacional, suas funções e como se relacionam
			     \item Análise crítica de soluções técnicas concorrentes para implementação de sistemas computacionais, considerando requisitos de desempenho e custo de um projeto
			     \item Diferenciação entre abordagens em software (uso de funções, tradução vs. interpretação, assembly vs. linguagem de nível mais alto) de acordo com requisitos de desempenho de execução e custo de implementação
			     \item Compreensão dos princípios de funcionamento das estratégias utilizadas para maximizar o desempenho de sistemas computacionais
	            \end{itemize}
         \end{itemize}
      \end{slide}

    \section[ slide = true]{Metodologia, ferramentas e avaliação}
      \begin{slide}[toc=]{Metodologia, ferramentas e avaliação}
         \begin{itemize}
	    \item Metodologia utilizada: aula invertida  
		    \begin{itemize}
			    \item Uma conferência web semanal (atividade síncrona): resolução de exercícios, aprofundamento do conteúdo
			    %\item Chat no SIGAA (atividade síncrona): plantão de dúvidas
			    \item Maior parte do estudo será feita de maneira guiada, mas assíncrona: leituras, vídeos, atividades diversas
		    \end{itemize}
	    \item Ferramentas usadas:
		    \begin{itemize}
			    \item Atividades síncronas: web conferência RNP% e SIGAA (Chat)
			    \item Atividades assíncronas: SIGAA (repositório e sistema de entrega)
			    \item Simulações, programação: Google Colaboratory (Python)
		    \end{itemize}

            \item Avaliação continuada
            \begin{itemize}
               \item Trabalhos 
               \item Tarefas 
	       \item Listas de exercício
	    \end{itemize}
	    %\item Cálculo da nota ($N$):
            %\begin{equation*}
            %   N=\frac{( \text{Tr} \times 4 + \text{Te} \times 6 )} {10}
            %\end{equation*}
          %\item Tabela de mapeamento:
          %  \begin{table}
          %     \centering
          %     \begin{tabular}{c|c}
          %        \hline\hline
          %        \textbf{Faixa} & \textbf{Conceito}\\
          %        \hline
          %        0 -- 4,9 & INS\\
          %        5,0 -- 6,9 & REG\\
          %        7,0 -- 8,9 & BOM\\
          %        9,0 -- 10,0 & EXC\\
          %        \hline\hline
          %     \end{tabular}
          %  \end{table}
	 \end{itemize}
      \end{slide}
      
      %\begin{slide}[toc=]{Datas dos testes escritos}
      %   \begin{itemize}
      %      \item Datas prováveis dos testes escritos:
      %      \begin{table}
      %         \centering
      %         \begin{tabular}{|l l|}
      %            \hline
      %            Teste 01: & 09/abril\\
      %            Teste 02: & 12/maio\\
      %            Teste 03: & 23/junho\\
      %            %Teste 04: & 05/dezembro\\
      %            %Substitutiva: & a definir\\
      %            \hline
      %         \end{tabular}
      %      \end{table}
      %   \end{itemize}
      %\end{slide}

\end{document}
